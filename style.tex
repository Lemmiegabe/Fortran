% Template.tex 
\documentclass[oneside]{article}
\usepackage{graphicx} % Required for inserting images

\pagestyle{empty}

\usepackage[margin=1.5in, marginparwidth=75pt]{geometry} 
\usepackage{amsmath,amsthm,amssymb,graphicx,mathtools,tikz,hyperref}
\DeclareMathOperator\arctanh{arctanh}
\usepackage{tikz}
\usepackage{lipsum}
\usepackage{lmodern}
\usepackage{tcolorbox}
\usepackage{slashed}
\usepackage{color,soul}
\usepackage[english]{babel} % To obtain English text with the blindtext package
\usepackage{blindtext}
\usepackage[T1]{fontenc}
\usepackage{xcolor}
\usepackage{lmodern}
\usepackage{listings}
\lstset{language=[77]Fortran,
  basicstyle=\ttfamily,
  keywordstyle=\color{black},
  commentstyle=\color{black},
  morecomment=[l]{!\ }% Comment only with space after !
}
\newcommand{\ts}{\textsuperscript}
\def\dbar{{\mathchar'26\mkern-12mu d}}
\usetikzlibrary{arrows.meta}
\newcommand\myarrow{%
  \tikz\draw[red,dashed,thick,-Triangle] (0,0) -- ++(0,-1.1);
  }
\newcommand{\n}{\mathbb{N}}
\newcommand{\z}{\mathbb{Z}}
\newcommand{\q}{\mathbb{Q}}
\newcommand{\cx}{\mathbb{C}}
\newcommand{\real}{\mathbb{R}}
\newcommand{\field}{\mathbb{F}}
\newcommand{\ita}[1]{\textit{#1}}
\newcommand{\com}[2]{#1\backslash#2}
\newcommand{\oneton}{\{1,2,3,...,n\}}
\newcommand\idea[1]{\begin{gather*}#1\end{gather*}}
\newcommand\ef{\ita{f} }
\newcommand\eff{\ita{f}}
\newcommand\proofs[1]{\begin{proof}#1\end{proof}}
\newcommand\inv[1]{#1^{-1}}
\newcommand\setb[1]{\{#1\}}
\newcommand\en{\ita{n }}
\newcommand{\vbrack}[1]{\langle #1\rangle}
\newcommand{\up}[1]{\textsuperscript{#1}}
\newcommand{\sub}[1]{\textsubscript{#1}}

\newenvironment{theorem}[2][Theorem]{\begin{trivlist}
\item[\hskip \labelsep {\bfseries #1}\hskip \labelsep {\bfseries #2.}]}{\end{trivlist}}
\newenvironment{lemma}[2][Lemma]{\begin{trivlist}
\item[\hskip \labelsep {\bfseries #1}\hskip \labelsep {\bfseries #2.}]}{\end{trivlist}}
\newenvironment{exercise}[2][Exercise]{\begin{trivlist}
\item[\hskip \labelsep {\bfseries #1}\hskip \labelsep {\bfseries #2.}]}{\end{trivlist}}
\newenvironment{reflection}[2][Reflection]{\begin{trivlist}
\item[\hskip \labelsep {\bfseries #1}\hskip \labelsep {\bfseries #2.}]}{\end{trivlist}}
\newenvironment{proposition}[2][Proposition]{\begin{trivlist}
\item[\hskip \labelsep {\bfseries #1}\hskip \labelsep {\bfseries #2.}]}{\end{trivlist}}
\newenvironment{corollary}[2][Corollary]{\begin{trivlist}
\item[\hskip \labelsep {\bfseries #1}\hskip \labelsep {\bfseries #2.}]}{\end{trivlist}}
 \hypersetup{colorlinks,linkcolor=blue}
 
 \begin{document}
 
 \date{07/09/2025}

 
\title{Style}

\author{Gabriel Lemmie}

\maketitle

\section*{Basics}

The text in Fortran has to follow a certain structure to be a valid Fortran program:

\begin{lstlisting}
	program circle
	real r, area 
c This program reads a real number r and prints 
c the area of a circle with radius r.
	write (*,*) 'Give radius r:'
	read (*,*) r
	area = 3.14159*r*r
	write(*,*) 'Area=', area 		
	stop 
	end		
\end{lstlisting}

The lines that begin with a "c" are \textit{comments} and have no purpose other than to make the program more readable for humans. Fortran is not case sensitive.

\subsection*{Program Organization}

A Fortran program generally consists of a main program (or driver) and possible several subprograms (procedures or subroutines). For now we will place all the statements in the main program; subprograms will be treated later. The structure of a main program is: 

\begin{lstlisting}
	program name 
	
	declarations
	
	statements
	
	stop
	end
\end{lstlisting}

The \texttt{stop} statement is optional and may seem superfluous since the program will stop when it reaches the end anyway, but it is recommended to always terminate a program with the \texttt{stop} statement to emphasize that the execution flow stops there. \\

\textbf{Note:} You cannot have a variable with the same name as the program.

\subsection*{Column Position Rules}

Fortran 77 is \textit{not} a free-format language, but has a very strict set of rules for how the source code should be formatted. The most important rules are the column position rules:

\begin{lstlisting}
 
Col.1:Blank, or a "c' or "*" for comments
Col.1-5:Statement label (optional) 
Col.6:Continuation of previous line (optional)
Col.7-72:Statements
Col.73-80:Sequence number (optional, rarely used today)
\end{lstlisting}
 Most lines in a Fortran 77 program star with 6 blanks and ends before column 72, i.e. only the statement field is used. 


\end{document}
